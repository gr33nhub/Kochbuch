\ifoot{\href{https://www.reddit.com/r/Cooking/comments/as9g4u/recipe_lemon_chicken_the_original_cantonese_style/}{https://www.reddit.com/r/}}
\section[Zitronenhaehnchen, original Kanton-Art]{\leafright\, Zitronenhaehnchen, original Kanton-Art \leafleft}
\begin{minipage}[t]{0.34\textwidth}
\vspace{0pt}
\fbox{\includegraphics[width=1\linewidth]{./Bilder/Zitronenhaehnchen_Kantonesischer_Stil.png}}
\vspace{0.5cm}

\begin{small}
\begin{tabular}{R{1.6cm} L{3.8cm} }
\multicolumn{2}{c}{\textbf{Zutaten für 2 Protionen}}\\ \toprule

250 g & Hühnerbrust, ohne Knochen, ohne Haut. (Alternativ: Entbeinte Hühner- oder Entenbrust)\\ \midrule[0.1mm]
1/2 & Zitrone, zesten und entsaften\\ \midrule[0.1mm]
1/2 TL & Salz\\ \midrule[0.1mm]
1/2 TL & Zucker\\ \midrule[0.1mm]
1/4 TL & liaojiu a.k.a. Shaoxing Wein\\ \midrule[0.1mm]
1 TL & leichte Sojasoße\\ \midrule[0.1mm]
\end{tabular}
\end{small}

\begin{small}
\begin{tabular}{R{1.6cm} L{3.8cm} }
\multicolumn{2}{c}{\textbf{Panade}}\\ \toprule

6 TL & cornstarch\\ \midrule[0.1mm]
1 & Ei\\ \midrule[0.1mm]
Sauce\\ \midrule[0.1mm]
50 mL & water\\ \midrule[0.1mm]
1 & tsp stock concentrate (Hühnersuppe)-or- homemade stock\\ \midrule[0.1mm]
1& /4 tsp salt\\ \midrule[0.1mm]
2 & tbsp sugar\\ \midrule[0.1mm]
1& /4 tsp instant custard powder (Puddingpulver) -or- milk powder\\ \midrule[0.1mm]
2 & cloves Galic. Gently smashed. Fried when making the sauce.\\ \midrule[0.1mm]
1 & inch Ginger\\ \bottomrule

\end{tabular}
\end{small}
Coating  for the chicken\\ \midrule[0.1mm]
cornstarch\\ \midrule[0.1mm]
& 1 egg\\ \midrule[0.1mm]
Sauce\\ \midrule[0.1mm]
50 mL & water\\ \midrule[0.1mm]
1 & tsp stock concentrate (Hühnersuppe)-or- homemade stock\\ \midrule[0.1mm]
1& /4 tsp salt\\ \midrule[0.1mm]
2 & tbsp sugar\\ \midrule[0.1mm]
1& /4 tsp instant custard powder (Puddingpulver) -or- milk powder\\ \midrule[0.1mm]
2 & cloves Galic. Gently smashed. Fried when making the sauce.\\ \midrule[0.1mm]
1 & inch Ginger\\ \bottomrule

\end{tabular}
\end{small}
\end{minipage}
\hfill
\begin{minipage}[t]{0.58\textwidth}
\vspace{0pt}
\subsection*{Zubereitung}
\begin{enumerate}[leftmargin=*, itemindent=14pt]
\item Butterfly the chicken breast, and cut grooves into it in a checkerboard
pattern. Slice into the chicken breast and unfold it. Slice little grooves into
the chicken about 1 cm apart, then flip the guy 90 degrees and do the same to
get a sort of checkered pattern. This will help the marinade go into the
chicken.

\item Add the marinade and marinate the chicken for at least 30 minutes.  Really
do a bang up job massaging that all into the chicken. 

\item Zest the lemon half, then cut into slices and squeeze out the juice. We’ll
keep these in the same bowl together as they’ll be added in the same time.

\item Smash the garlic cloves and ginger; mix together the ingredients for the
sauce; mix together your slurry. Keep the slurry separate from the rest of the
sauce as it’ll go in at the very end.

\item Crack an egg and beat it until no stray strands of egg white remain. Mix
in 6 tbsp of the cornstarch, stirring until a smooth batter forms.  In a
separate plate, pour out enough cornstarch to do a proper coating.  Apologies
for the lack of measuring with regards to the loose cornstarch – I’d venture
you’d want at least a cup.

\item Add a tbsp of cornstarch to the chicken, then coat it completely with the
batter, and finally flip both sides onto the dry cornstarch to coat. The initial
bit of cornstarch is to help the batter stick better to the chicken.

\item In a wok with a couple cups of oil, heat it up until about 180C. Toss in
the chicken and fry for about five minutes at 170C, flipping occasionally. As
soon as you toss in the chicken it’ll lower the temperature, which’s why we
overshot things at first. We flipped twice – once at the 2 minute mark, and
again at the 4 minute mark. If you’ve got a bit peaking out when you flip, spoon
over some hot oil with your spatula as it cooks. Once the chicken’s lightly
golden brown, take it out.

\item Heat the oil up until 200C, then dip the cutlet in again for 30 seconds.
Transfer over to a paper towel lined plate. The second fry is to make the
coating slightly crispier.

\item In a separate pot over medium flame, add in a ½ tbsp of oil and fry the
garlic and ginger until fragrant, ~30 seconds. Swirl in the liaojiu wine, then
go in with the sauce. Let it boil for about a minute, then remove the garlic and
the ginger. Add in the lemon juice and zest, simmer for ~15 seconds, then go in
with the slurry. Let it boil together for ~30 seconds til thickened. After all
that, set the sauce aside.

\item Chop the cutlet into roughly 1.5 inch by 2 inch pieces.  Toss on a plate
and smother with your lemon sauce.

\item Garnish with either the lemon rind or extra lemon slices, and optionally a
touch of cilantro. If you had no other plan for the other half of the lemon,
using lemon slices adds a nice fragrance to the dish. The cilantro’s totally
superfluous, but hey, green things look nice.
\end{enumerate}
\end{minipage}
\vfill
\decothreeright \, \textbf{Arbeitszeit:} ca. 1 Std. / \textbf{Koch-Backzeit:} ca. 2 Std. /\textbf{Schwierigkeitsgrad:} pfiffig \decothreeleft \hfill \\ Bewertung:  \Circle  \Circle \Circle \Circle \Circle
