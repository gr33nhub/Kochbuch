\ifoot{\href{https://missboulette.wordpress.com/2010/07/26/4-schritte-zum-ideal-gerosteten-sesam/}{https://missboulette.wordpress.com/2010/07/26/4-schritte-zum-ideal-gerosteten-sesam/}}
\section[4 Schritte zum ideal gerösteten Sesam]{\leafright\, 4 Schritte zum ideal gerösteten Sesam \leafleft}
\label{sec:4 Schritte zum ideal gerösteten Sesam}

\begin{minipage}[t]{0.34\textwidth}
\vspace{0pt}\fbox{\includegraphics[width=1\linewidth]{./Bilder/ideal_geroesteter_Sesam.png}}
\vspace{0.5cm}

\begin{small}
\begin{tabular}{R{1.6cm} L{3.8cm} }
\multicolumn{2}{c}{\textbf{Zutaten für 1 Portion}}\\ \toprule
etwa 1 & kleine Kaffetasse
\end{tabular}
\end{small}
\end{minipage}
\hfill
\begin{minipage}[t]{0.58\textwidth}
\vspace{0pt}
\subsection*{Zubereitung}
\begin{enumerate}[leftmargin=*, itemindent=14pt]

\item Erhitze eine beschichtete Pfanne bei mittlerer Hitze auf dem Herd.

\item Den Sesam kurz unter kaltem Wasser abwaschen und im Sieb oder auf einem Tuch abtropfen lassen. Gib den noch feuchten Sesam in die Pfanne. Extra Öl brauchst Du zum Rösten nicht.

\item Röste die Kerne unter ständigem Rühren. Nach etwa 2 bis 3 Minuten beginnen die Kerne zu knistern und durch die Pfanne zu hüpfen.

\item Nimm ab dann immer mal wieder Körner zwischen die Finger und versuche den Sesam zwischen ihnen zu zerreiben. Gelingt das einfach, ist der Sesam fertig geröstet.

\item Fülle den gerösteten Sesam in eine Schüssel um und lasse ihn dort auskühlen. In der heißen Pfanne würde er anderenfalls weiterrösten und verbrennen.

\end{enumerate}

Beim Rösten geht es um Sekunden! Die ideale Röststufe ist kurz vor dem verbrennen der Samen erreicht. Sobald der Zenit erreicht ist fällt die Qualität rasant ab. Erkennen kann man das an übermäßiger Rauchentwicklung, d.h. Fett tritt bereits aus den Körnern aus und verbrennt. Die Körner sind ölig, einige bereits geplatzt. Diese Körner würden nur wenige Tage gut, danach schnell unangenehm ranzig schmecken.

Wird der Sesam zum idealen Zeitpunkt herausgenommen, kann er, trocken, luftdicht und dukel gelagert, einige Monate gut überstehen.

\end{minipage}
\vfill
\decothreeright \, \textbf{Arbeitszeit:} wenige Minuten	 / \textbf{Schwierigkeitsgrad:} simpel	 / \decothreeleft \hfill Bewertung: \Circle  \Circle \Circle  \Circle \Circle