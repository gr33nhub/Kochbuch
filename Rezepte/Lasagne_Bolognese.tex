\ifoot{\href{http://www.chefkoch.de/rezepte/1112181217260303/Lasagne-Bolognese.html}{http://www.chefkoch.de/rezepte/1112181217260303/Lasagne-Bolognese.html}}
\section[Lasagne Bolognese]{\leafright\, Lasagne Bolognese \leafleft}
\begin{minipage}[t]{0.34\textwidth}
\vspace{0pt}
\fbox{\includegraphics[width=1\linewidth]{./Bilder/lasagne-bolognese.png}}
\vspace{0.5cm}

\begin{small}
\begin{tabular}{R{1.6cm} L{3.8cm} }
\multicolumn{2}{c}{\textbf{Zutaten für 4 Portionen }}\\ \toprule
8 	& Lasagneplatte(n), gekocht\\ \midrule[0.1mm]
600 g&	 Hackfleisch\\ \midrule[0.1mm]
3 EL&	 Olivenöl\\ \midrule[0.1mm]
1 große	& Möhre(n)\\ \midrule[0.1mm]
1 Stück	& Sellerie\\ \midrule[0.1mm]
1 große	& Zwiebel(n)\\ \midrule[0.1mm]
2 Zehe/n&	 Knoblauch\\ \midrule[0.1mm]
200 ml	& Weißwein\\ \midrule[0.1mm]
500 ml&	 Brühe\\ \midrule[0.1mm]
3 EL	& Tomatenmark\\ \midrule[0.1mm]
 	& Salz und Pfeffer, Zucker\\ \midrule[0.1mm]
2 EL& Butter\\ \midrule[0.1mm]
3 EL& Mehl gestr.\\ \midrule[0.1mm]
500 ml&	 Milch\\ \midrule[0.1mm]
 	 &Muskat\\ \midrule[0.1mm]
100 g&	 Parmesan, frisch gerieben\\ \bottomrule
\end{tabular}
\end{small}
\end{minipage}
\hfill
\begin{minipage}[t]{0.58\textwidth}
\vspace{0pt}
\subsection*{Zubereitung}
\begin{enumerate}[leftmargin=*, itemindent=14pt]
\item Möhre, Sellerie, Zwiebel + Knoblauch putzen und in Würfel schneiden. Das Öl erhitzen, Würfel gut anbraten und wieder aus der Pfanne nehmen. 

\item Nun das Hackfleisch zur Hälfte zufügen und krümelig bzw. kross anbraten. Wieder aus der Pfanne nehmen und das restliche Hack anbraten. Das bereits gebratene Hack und Gemüse wieder zufügen, mit dem Wein ablöschen und fast verkochen lassen. Tomatenmark dazu geben, etwas angehen lassen und mit der Brühe auffüllen. Mit Deckel sämig einkochen lassen. Mit Salz, Pfeffer und einer Prise Zucker würzen.

\item Für die Bechamel die Butter in einem Topf zerlassen. Das Mehl einrühren und kurz anschwitzen, dann unter Rühren nach und nach die Milch zugießen. Mit Muskat und Salz abschmecken und einmal aufkochen lassen.

\item Auflaufform fetten und mit Lasagneblättern belegen. Einige Löffel Bechamel darauf verteilen, mit etwas Parmesan bestreuen und etwas von der Bolognese darüber geben. So weiter schichten, bis alle Zutaten verbraucht sind. Die oberste Schicht sollte aus der Bechamelsauce bestehen, die gleichmäßig mit dem Parmesan bestreut wird.

\item Im vorgeheizten Backofen bei 180°C ca. 20 Minuten überbacken. Bevor die Lasagne portioniert wird, einige Minuten ruhen lassen. Guten Appetit!
\end{enumerate}
\end{minipage}
\vfill
\decothreeright \, \textbf{Arbeitszeit:} ca. 1 Std. 15 Min. / \textbf{Schwierigkeitsgrad:} normal \decothreeleft \hfill Bewertung:  \CIRCLE  \CIRCLE \CIRCLE \CIRCLE \CIRCLE 