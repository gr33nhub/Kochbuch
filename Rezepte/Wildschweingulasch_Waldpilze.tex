\ifoot{\href{http://www.chefkoch.de/rezepte/1832561297156839/Wildschweingulasch-mit-Waldpilzen.html}{http://www.chefkoch.de/rezepte/Wildschweingulasch-mit-Waldpilzen.html}}
\section[Wildschweingulasch mit Waldpilzen]{\leafright\, Wildschweingulasch mit Waldpilzen \,\leafleft}
\begin{minipage}[t]{0.34\textwidth}
\vspace{0pt}\fbox{\includegraphics[width=1.0\textwidth]{./Bilder/wildschweingulasch-waldpilzen.png}}
\vspace{0.5cm}

\begin{small}
\begin{tabular}{R{1.7cm} L{3.7cm} }
\multicolumn{2}{c}{\textbf{Zutaten für 4 Portionen }}\\ \toprule
500 g&	 Gulasch vom Wildschwein\\ \midrule[0.1mm]
2 &	 Zwiebeln\\ \midrule[0.1mm]
2 EL&	 Tomatenmark\\ \midrule[0.1mm]
250 ml&	 Rotwein, trocken\\ \midrule[0.1mm]
400 ml&	 Wildfond\\ \midrule[0.1mm]
400 ml&	 Rinderbrühe, (Instant)\\ \midrule[0.1mm]
500 g&	 Pilze, gemischt, TK; auf gute Qualität achten\\ \midrule[0.1mm]
2 &	 Wacholderbeeren\\ \midrule[0.1mm]
1 	& Piment\\ \midrule[0.1mm]
2 	& Lorbeerblätter\\ \midrule[0.1mm]
3/4 Becher&	 Crème fraîche\\ \midrule[0.1mm]
 &	 Salz und Pfeffer\\ \bottomrule
\end{tabular}
\end{small}
\end{minipage}
\hfill
\begin{minipage}[t]{0.58\textwidth}
\vspace{0pt}
\subsection*{Zubereitung}
\begin{enumerate}[leftmargin=*, itemindent=14pt]
\item Das Gulasch waschen, trocken tupfen und mit Salz und Pfeffer würzen. Die Zwiebeln in feine Ringe schneiden.

\item Das Gulasch in etwas Öl oder Schmalz rundherum anbraten bis er eine schöne Farbe hat. Dann herausnehmen und abgedeckt zur Seite stellen. Die Zwiebelringe im gleichen Fett anbraten bis sie Farbe genommen haben. Die 2 Löffel Tomatenmark dazugeben und kurz mitbraten. Mit dem Rotwein ablöschen und aufkochen, den Wildfond angießen und aufkochen und dann mit der Brühe ergänzen. Wacholderbeeren, Lorbeerblätter und Piment dazugeben und mit geschlossem Deckel 1 Stunde und 45 Minuten schmoren lassen. Gelegentlich umrühren.

\item Ca. 15 Minuten bevor die Zeit um ist, die Waldpilze tiefgefroren in eine zweite Pfanne geben und auf großer Hitze braten bis das austretende Wasser vollständig verdampft ist. Mit Salz und Pfeffer würzen, mit etwas Wasser oder Brühe ablöschen und zu dem Wildschweingulasch geben.

\item Zusammen nochmal 10 Minuten weiterköcheln. Dann Creme fraiche einrühren, Lorbeerblätter, Wacholderbeeren und Piment entfernen und auf gewünschte Konsistenz eindicken. Falls notwendig (in der Regel nicht) mit Salz und Pfeffer würzen.

\end{enumerate}
Dazu passen Spätzle und evtl. Rotkraut.
\end{minipage}
\vfill
\decothreeright \, \textbf{Arbeitszeit:} 15 Min. / \textbf{Schwierigkeitsgrad:} normal \decothreeleft \hfill Bewertung:  \CIRCLE \CIRCLE \CIRCLE \CIRCLE \CIRCLE