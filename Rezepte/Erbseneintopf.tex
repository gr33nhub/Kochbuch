\ifoot{\href{http://www.chefkoch.de/rezepte/1613999850871/Erbseneintopf.html}{http://www.chefkoch.de/rezepte/1613999850871/Erbseneintopf.html}}
\section[Erbseneintopf]{\leafright\, Erbseneintopf \leafleft}
\begin{minipage}[t]{0.34\textwidth}
\vspace{0pt}\fbox{\includegraphics[width=1\linewidth]{./Bilder/erbseneintopf.png}}
\vspace{0.5cm}

\begin{small}
\begin{tabular}{R{1.6 cm} L{3.8cm} }
\multicolumn{2}{c}{\textbf{Zutaten für 4 Portionen}}\\ \toprule
250 g&	 Erbsen, getrocknete\\ \midrule[0.1mm]
80 g&	 Speck, durchwachsen, gut geräuchert\\ \midrule[0.1mm]
1&	 Zwiebel, m.-groß\\ \midrule[0.1mm]
150 g&	 Karotten\\ \midrule[0.1mm]
100 g&	 Knollensellerie\\ \midrule[0.1mm]
300 g&	 Kartoffel(n), mehlig kochend\\ \midrule[0.1mm]
1 TL&	 Majoran, gerebelt, ca.\\ \midrule[0.1mm]
 	& Salz und Pfeffer, weiß aus der Mühle\\ \midrule[0.1mm]
4 	& Würstchen, Wiener\\ \midrule[0.1mm]
1 EL&	 Petersilie, gehackt\\ \bottomrule

\end{tabular}
\end{small}
\end{minipage}
\hfill
\begin{minipage}[t]{0.58\textwidth}
\vspace{0pt}
\subsection*{Zubereitung}
\begin{enumerate}[leftmargin=*, itemindent=14pt]

\item Die Erbsen waschen und über Nacht in ca. 1,5 L Wasser einweichen. 

\item Zwiebeln klein schneiden. Karotten, Sellerie und Kartoffeln schälen/putzen und in Würfelchen schneiden. Speck würfeln.

\item Die Zwiebel zusammen mit dem Speck in einem Topf auslassen. Karotten- und Selleriewürfel dazu geben und kurz anbraten lassen. Erbsen mit dem Einweichwasser dazugeben und alles ca. 1 Stunde köcheln lassen, dabei ab und an umrühren, damit nichts ansetzt. 

\item Dann die Kartoffeln und den Majoran zugeben, weiterkochen. Sind die Kartoffeln gar, den Eintopf würzen und die Würstchen darin heiß werden lassen. 

\item Wer möchte, kann den Eintopf auch, natürlich ohne die Würstchen, pürieren. Zum Schluss die Petersilie darüber streuen.

\end{enumerate}

Dazu schmeckt eine dicke Scheibe frisches kräftiges Bauernbrot.
\end{minipage}
\vfill
\decothreeright \, \textbf{Arbeitszeit:} ca. 20 Min.	 / \textbf{Schwierigkeitsgrad:} normal	 / \decothreeleft \hfill Bewertung: \CIRCLE \CIRCLE \CIRCLE \CIRCLE \Circle