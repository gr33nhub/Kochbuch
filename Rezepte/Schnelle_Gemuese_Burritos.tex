\ifoot{\href{https://www.chefkoch.de/rezepte/420751132752442/Schnelle-Gemuese-Burritos.html}{https://www.chefkoch.de/rezepte/420751132752442/Schnelle-Gemuese-Burritos.html}}

\section[Schnelle Gemüse - Burritos]{\leafright\, Schnelle Gemüse - Burritos \leafleft}
\begin{minipage}[t]{0.34\textwidth}
\vspace{0pt}
\fbox{\includegraphics[width=1\linewidth]{./Bilder/schnelle_gemuese_burritos.png}}
\vspace{0.5cm}

\begin{small}
\begin{tabular}{R{1.6cm} L{3.8cm} }
\multicolumn{2}{c}{\textbf{Zutaten für 4 Portionen }}\\ \toprule 

8 & 	Tortilla(s)\\ \midrule[0.1mm]
1  &	Zwiebel(n)\\ \midrule[0.1mm]
1 Zehe/n &	Knoblauch\\ \midrule[0.1mm]
2  &	Karotte(n)\\ \midrule[0.1mm]
1  &	Paprikaschote(n), rote\\ \midrule[0.1mm]
1  &	Paprikaschote(n), gelbe\\ \midrule[0.1mm]
1  &	Paprikaschote(n), grüne\\ \midrule[0.1mm]
1 Dose/n &	Mais\\ \midrule[0.1mm]
1 Dose/n &	Kidneybohnen\\ \midrule[0.1mm]
1 Becher &	Crème fraîche\\ \midrule[0.1mm]
  &	Salz und Pfeffer und Cayennepfeffer\\ \midrule[0.1mm]
  &	Oregano \\ \bottomrule

\end{tabular}
\end{small}

\end{minipage}
\hfill
\begin{minipage}[t]{0.58\textwidth}
\vspace{0pt}
\subsection*{Zubereitung}
\begin{enumerate}[leftmargin=*, itemindent=14pt]

    \item Die Tortillas nach Belieben leicht anbraten.

    \item Die Zwiebel und die Knoblauchzehe schälen und fein hacken, die Karotte in Scheiben schneiden und alles in etwas Öl in einer großen Pfanne andünsten.

    \item Währenddessen die Paprikaschoten entkernen, würfeln und dann etwas mitdünsten.

    \item Kidneybohnen und Mais abgießen und unter das Gemüse mengen.

    \item Auf jede Tortilla nun etwa 2-3 El Gemüse und 2-3 Tl Creme fraiche geben, die Seiten darüber zusammenschlagen und die Unterkante hochklappen. Am besten in Alufolie wickeln, damit man sie sauber essen kann.

    \item Zum warm halten am besten den Ofen kurz anheizen und die Burritos darin bei geschlossener Tür bis zum Verzehr lagern (aber bitte nicht länger als 20 min!)


    \leafNE\, Überbacken

    lternativ können die Burritos auch in eine Auflaufform geschichtet werden und bei 200°C für 12 bis 15 min mit Käse überbacken werden. 

\end{enumerate}
\end{minipage}
\vfill
\decothreeright \, \textbf{Arbeitszeit:} ca. 20 Min. / \textbf{Schwierigkeitsgrad:} normal \decothreeleft \hfill Bewertung: \CIRCLE \CIRCLE \CIRCLE \CIRCLE \LEFTcircle 