\ifoot{\href{http://www.chefkoch.de/rezepte/1754281285077184/Roemische-Zucchini.html}{http://www.chefkoch.de/rezepte/1754281285077184/Roemische-Zucchini.html}}
\section[Römische Zucchini]{\leafright\, Römische Zucchini \,\leafleft}
\begin{minipage}[t]{0.34\textwidth}
\vspace{0pt}
\fbox{\includegraphics[width=1\linewidth]{./Bilder/roemische-zucchini.png}}
\vspace{0.5cm}

\begin{small}
\begin{tabular}{R{1.6cm} L{3.8cm} }
\multicolumn{2}{c}{\textbf{Zutaten für 4 Portionen }}\\ \toprule
50 g&	 Speck, gewürfelt\\ \midrule[0.1mm]
10 ml&	 Olivenöl\\ \midrule[0.1mm]
1 &	 Zwiebel, gehackt\\ \midrule[0.1mm]
1 &	 Knoblauchzehe, gehackt\\ \midrule[0.1mm]
500 g&	 Tomaten, sehr saftig\\ \midrule[0.1mm]
1 EL&	 Oregano, gehackt\\ \midrule[0.1mm]
2 EL&	 Petersilie, gehackt\\ \midrule[0.1mm]
1 EL&	 Schnittlauch, gehackt\\ \midrule[0.1mm]
 	& Salz\\ \midrule[0.1mm]
60 g&	 Sahne\\ \midrule[0.1mm]
500 g&	 Zucchini, kleine\\ \midrule[0.1mm]
2 EL&	 Semmelbrösel\\ \midrule[0.1mm]
40 g&	 Parmesan\\ \midrule[0.1mm]
60 g&	 Sahne\\ \bottomrule
\end{tabular}
\end{small}
\end{minipage}
\hfill
\begin{minipage}[t]{0.58\textwidth}
\vspace{0pt}
\subsection*{Zubereitung}
\begin{enumerate}[leftmargin=*, itemindent=14pt]
\item Speck anbraten. Zwiebeln und Knoblauch zum Speck zugeben. Tomaten enthäuten, grob zerkleinern. Mit Oregano, Salz und Pfeffer würzen. Sahne hinzugeben. Köcheln lassen, bis eine dickliche Soße entsteht.
\item Von den Zucchinis die Enden abschneiden, längs halbieren und mit Salz bestreuen.

\item Semmelbrösel, Petersilie, Schnittlauch, (ggf. auch Dill), Parmesan und Sahne zu einer dicklichen Masse verrühren.

\item Tomatensoße in eine Auflaufform geben, die Zucchini darauf setzen. Auf die Zucchinihälften die Masse streichen. Bei 225-250°C ca. 20-25 Minuten im Backofen überbacken.
\end{enumerate}
Dazu passt sehr gut frisches Baguette.\\
Schmeckt auch super ohne Sahne.
\end{minipage}
\vfill
\decothreeright \, \textbf{Arbeitszeit:} 30 Min. / \textbf{Schwierigkeitsgrad:} normal \decothreeleft \hfill Bewertung:  \CIRCLE \CIRCLE \CIRCLE \CIRCLE  \LEFTcircle