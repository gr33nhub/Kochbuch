\ifoot{\href{https://www.kochbar.de/rezept/265348/Beilage-Sauerkraut-selbermachen.html/}{https://www.kochbar.de/rezept/265348/Beilage-Sauerkraut-selbermachen.html/}}
\section[Sauerkraut]{\leafright\, Sauerkraut \leafleft}
\begin{minipage}[t]{0.34\textwidth}
\vspace{0pt}
\fbox{\includegraphics[width=1\linewidth]{./Bilder/sauerkraut-selbermachen-rezept.png}}
\vspace{0.5cm}

\begin{small}
\begin{tabular}{R{1.6cm} L{3.8cm} }
\multicolumn{2}{c}{\textbf{Zutaten für 10 Liter }}\\ \toprule
8kg & Weißkraut / Kohlkopf\\ \midrule[0.1mm]
120g& Salz (15 g pro kg Kraut\\ \bottomrule
\end{tabular}
\end{small}
\end{minipage}
\hfill
\begin{minipage}[t]{0.58\textwidth}
\vspace{0pt}
\subsection*{Zubereitung}
\begin{enumerate}[leftmargin=*, itemindent=14pt]


\item Tontopf mit heißem Wasser ausbrühen und dann an der Luft trocknen. Wenn der Tontopf trocken ist, dann legen Sie ihn mit sauberen Weißkohl-Blättern aus. Dadurch bleibt das Sauerkraut von der Farbe her schön hell. Anschließend beginnen Sie nun mit dem eigentlichen Sauerkraut machen.

\item Zuerst entfernen Sie die äußeren grünen Blätter schneiden den Kohlkopf in der Mitte durch und entfernen den dicken Strunk.

\item Nun geht es mit dem Krauthobel weiter. In einer großen Schüssel oder Kunstoffbadewanne sammeln Sie das geraspelte Kraut. Dieses durchmengen Sie im nächsten Schritt mit Salz, das ist wichtig um zu Beginn der Gärung unerwünschte Mikroben fernzuhalten. Ca. 15 Gramm Salz auf 1kg Kohl.

\item Dieses Gemenge kräftig mit den Händen quetschen oder mit den Fäusten stampfen bis sich ordentlich Brühe bildet. Dann in den Tontopf, ca. ¾ voll, einfüllen. Festdrücken, bis ca. 2 cm Brühe über dem Kraut stehen. Nun werden einige Krautblätter als Abschluß aufgelegt.

\item Danach werden auf das Sauerkraut die Abschlußsteine gelegt und eventuell noch mit einem faustgroßen Kieselstein beschwert. Dies ist nötig, damit das Sauerkraut immer in der Lake bleibt und nicht aufquellen kann. Der Tontopf hat einen zusätzlichen Rand, in dem der Deckel liegt. Dort wird Wasser eingefüllt. So können von außen keine Keime eindringen. Die Gase können darüber leicht entweichen

\item 14 Tage bis 3 Wochen bleibt der Topf nun in einem etwas wärmeren Raum stehen, dann erst kommt er in den Keller. Nach weiteren 1 – 2 Wochen kann man ja mal probieren, ob es schon genügend gesäuert hat.
\end{enumerate}

TIPP: In die Tontopfrinne eine starke Salzlake füllen. Es können sich ansonsten in der Rinne Schleimbakterien bilden. Wenn der Topf in einen kühleren Raum gebracht wird, erfolgt beim Temperaturangleich zeitweise ein Sog nach innen. Somit könnten Schleimbakterien ins Kraut gelangen. Meinen ersten Ansatz dieses Jahr konnte ich vermutlich aus diesem Grund entsorgen.
    
\end{minipage}
\vfill
\decothreeright \, \textbf{Arbeitszeit:} - / \textbf{Schwierigkeitsgrad:} leicht \decothreeleft \hfill Bewertung: \Circle  \Circle \Circle  \Circle \Circle