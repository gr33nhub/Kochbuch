\ifoot{\href{http://www.chefkoch.de/rezepte/341201118049854/Pikante-Reistorte.html}{http://www.chefkoch.de/rezepte/341201118049854/Pikante-Reistorte.html}}
\section[Pikante Reistorte]{\leafright\, Pikante Reistorte \leafleft}
\begin{minipage}[t]{0.33\textwidth}
\vspace{0pt}
\fbox{\includegraphics[width=1\linewidth]{./Bilder/pikante_reistorte.png}}
\vspace{0.5cm}

\begin{tabular}{r p{3.6cm} }
\multicolumn{2}{c}{\textbf{Zutaten für 4 Portionen }}\\ \toprule
400 g& Hähnchenbrustfilet (oder Putenschnitzel)\\ \midrule
2 &	 Paprikaschote(n)\\ \midrule
2 &	 Tomate(n)\\ \midrule
1 &	 Zwiebel(n)\\ \midrule
4 &	 Ei(er)\\ \midrule
1 Becher& Crème fraîche (150g)\\ \midrule
200 g&	 geriebener Käse (Emmentaler / Gouda)\\ \midrule
250 g&	 Reis\\ \midrule
 	& Salz und Pfeffer\\ \midrule
 	& Tabasco\\ \midrule
 	& Sonnenblumenöl\\ \bottomrule
\end{tabular}

\end{minipage}
\hfill
\begin{minipage}[t]{0.60\textwidth}
\vspace{0pt}
\subsection*{Zubereitung}
Hähnchenbrust in kleine Würfel schneiden, Paprika und Zwiebel würfeln. Die Hähnchenbrust in Sonnenblumenöl anbraten, Zwiebeln hinzugeben und glasig werden lassen. Die Paprikawürfel kurz mitbraten (sollen noch gut Biss haben). Alles mit Salz, Pfeffer, Tabasco gut würzen. Aus dem Topf nehmen.\\

In dem übriggebliebenen Öl den Reis glasig dünsten, Wasser hinzugeben (1 Teil Reis / 2 Teile Wasser), salzen und quellen lassen (sollte noch etwas Biss haben).\\

In der Zwischenzeit Eier mit Creme Fraiche und 150 g Käse vermischen. Dann Hähnchen, Paprika und Reis hinzugeben. Gut mischen und noch mal kräftig mit Salz, Pfeffer und Tabasco abschmecken. In eine Springform füllen und den restlichen Käse darüber streuen. Tomaten in Scheiben schneiden und auf der Reismasse verteilen. Bei 200 Grad 20 Minuten backen.\\

Schmeckt besonders gut, wenn man die Torte mit einer fruchtigen Tomatensoße serviert. Ist gut vorzubereiten und schmeckt auch aufgewärmt gut.
\end{minipage}
\vfill
\decothreeright \, \textbf{Arbeitszeit:} ca. 30 Min. / \textbf{Schwierigkeitsgrad:} normal \decothreeleft \hfill Bewertung: \CIRCLE \LEFTcircle  \Circle  \Circle \Circle