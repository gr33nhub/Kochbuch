\ifoot{\href{http://www.chefkoch.de/rezepte/1200081225877998/Bunte-Frittata.html}{http://www.chefkoch.de/rezepte/1200081225877998/Bunte-Frittata.html}}
\section[Bunte Frittata]{\leafright\, Bunte Frittata \leafleft}
\begin{minipage}[t]{0.34\textwidth}
\vspace{0pt}\fbox{\includegraphics[width=1\linewidth]{./Bilder/bunte_frittata.png}}
\vspace{0.5cm}

\begin{small}
\begin{tabular}{R{1.6cm} L{3.8cm} }
\multicolumn{2}{c}{\textbf{Eine Quicheform 24 cm }}\\ \toprule
500 g&	 Karotten\\ \midrule[0.1mm]
250 g&	 Brokkoli\\ \midrule[0.1mm]
4 &	 Eier\\ \midrule[0.1mm]
75 g&	 Sauerrahm\\ \midrule[0.1mm]
1/2 TL& Salz (gestr.)\\ \midrule[0.1mm]
1 TL& 	 Currypulver(gestr.)\\ \midrule[0.1mm]
 etwas&	 Kreuzkümmel, evtl.\\ \midrule[0.1mm]
2 EL&	 Sesam, evtl.\\ \midrule[0.1mm]
 etwas&	 Fett für die Form\\ \midrule[0.1mm]
 evtl.&	 Käse\\ \midrule[0.1mm]
 evtl.&	 Tomate(n)\\ \midrule[0.1mm]
 evtl.&	 Erbsen\\ \midrule[0.1mm]
 evtl.&	 Zucchini\\ \midrule[0.1mm]
 evtl.&	 Kochschinken\\ \bottomrule
\end{tabular}
\end{small}

\end{minipage}
\hfill
\begin{minipage}[t]{0.58\textwidth}
\vspace{0pt}
\subsection*{Zubereitung}
\begin{enumerate}[leftmargin=*, itemindent=14pt]
\item Den Backofen auf 180 °C vorheizen. Die Karotten waschen und schälen. Dünnere längs halbieren, Dickere vierteln. Vom Brokkoli den Strunk abschneiden.\\ 

\item Die Eier mit Sauerrahm verquirlen und würzen. 1,5 EL Sesam dazugeben. Die Form (Quicheform 24 cm) fetten, die Eiermasse einfüllen. Das Gemüse darauf verteilen. Mit Sesam bestreuen und im Backofen (Mitte, Umluft 160 °C) in 35 Minuten stocken lassen. \\

\item Die Frittata in Stücke schneiden und anrichten. Als Beilage passen Brot oder Kartoffeln. Reste kann man auch kalt für die Jause verwenden. \\
\end{enumerate}
Varianten: Gemüse nach Saison variieren, z. B.: 150 g Erbsen und 200 g Cocktailtomaten, 1 Paprikaschote und 2 Zucchini (ca. 160 g), Mozzarella und Tomaten, geraspeltem Käse, 2 Scheiben Kochschinken würfeln und unter die Eiermasse mischen.
\end{minipage}
\vfill
\decothreeright \, \textbf{Arbeitszeit:} ca. 20 Min.	 / \textbf{Schwierigkeitsgrad:} simpel	 / \decothreeleft \hfill Bewertung: \LEFTcircle  \Circle \Circle  \Circle \Circle