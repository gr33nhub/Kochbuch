\ifoot{}
\section[Haferflockenkekse mit Erdnüsse]{\leafright\, Haferflockenkekse mit Erdnüsse\,\leafleft}
\begin{minipage}[t]{0.34\textwidth}
\vspace{0pt}
\fbox{\includegraphics[width=1\linewidth]{./Bilder/roemische-zucchini.png}}
\vspace{0.5cm}

\begin{small}
\begin{tabular}{R{1.6cm} L{3.8cm} }
\multicolumn{2}{c}{\textbf{Zutaten für 4 Portionen }}\\ \toprule
150 g&	 Mehl\\ \midrule[0.1mm]
200 g&	 Zucker, braun\\ \midrule[0.1mm]
80 g &	 Haferflocken\\ \midrule[0.1mm]
200 g &	 Erdnüsse, geröstet ohne Salz\\ \midrule[0.1mm]
1 &	 Ei\\ \midrule[0.1mm]
1 Tl&	 Zimt\\ \midrule[0.1mm]
1 Tl&	 Backpulver\\ \midrule[0.1mm]
1/2 Tl&	 Zimt\\ \midrule[0.1mm]
25 g&	 Ingwer\\ \midrule[0.1mm]
1-2 El&	 Milch\\ \midrule[0.1mm]
3 El&	 Öl\\ \bottomrule
\end{tabular}
\end{small}
\end{minipage}
\hfill
\begin{minipage}[t]{0.58\textwidth}
\vspace{0pt}
\subsection*{Zubereitung}
\begin{enumerate}[leftmargin=*, itemindent=14pt]
\item Backofen vorheizen auf 200°C Umluft, den Ingwer klein schneiden.
\item Ei, Zucker und Öl schaumig rühren. Der Zucker sollte sich auflösen.
\item Haferflocken, Mehl und Backpulver dazugeben und gut durchmengen.
\item Erdnüsse, Ingwer, Milch und Zimt unterrühren. Wenn der Teig zu fest ist noch etwas Milch hinzugeben.
\item Kleine Häufchen auf dem mit Backpapier ausgelegten Backblech machen und ca. 8-10 Minuten (14-15 Min bei Oberhitze) backen.
\end{enumerate}
\end{minipage}
\vfill
\decothreeright \, \textbf{Arbeitszeit:} 30 Min. / \textbf{Schwierigkeitsgrad:} einfach \decothreeleft \hfill Bewertung:  \CIRCLE \CIRCLE \CIRCLE \CIRCLE  \LEFTcircle