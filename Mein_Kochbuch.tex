\documentclass[
	a4paper, % Papierformat
	BCOR=20mm,
	ngerman, % für Umlaute, Silbentrennung etc.
	titlepage, % es wird eine Titelseite verwendet
	parskip=half, % Abstand zwischen Absätzen (halbe Zeile)
	final % Status des Dokuments (final/draft)
]{scrreprt}
\usepackage{scrhack}

%\overfullrule=5pt %visualisiere volle/leere Boxen

\usepackage{lmodern} % bessere Fonts
\usepackage[T1]{fontenc} %Vektorschriften
\usepackage[utf8]{inputenc} %üäöß usw richtg erkennen...
\usepackage[ngerman]{babel} %english,french....
\usepackage[babel]{microtype} %verbessertes schriftbild
\usepackage{relsize} % Schriftgröße relativ festlegen
\usepackage{subfig,wrapfig}
% zum Einbinden von Programmcode

\usepackage{kpfonts} %http://sunsite.informatik.rwth-aachen.de/ftp/pub/mirror/ctan/fonts/kpfonts/doc/kpfonts.pdf
%\usepackage{mathptmx} % für Schrift mit Serifen
%\usepackage[scaled=.90]{helvet} % für Schrift ohne Serifen
%\usepackage{courier} % für Schrift mit konstanter Breite


\usepackage{calc}% http://ctan.org/pkg/calc
\usepackage{enumitem} %erweiterte Listen

\usepackage{fourier-orns} %schnörkel
\usepackage{wasysym} %erweiterte Symbole

\usepackage{lastpage}

% -----------------------------------------------
\usepackage{listings}
\usepackage{xcolor}

\usepackage{graphicx} % um Bilder mit \graphics einbinden zu können
\usepackage[dvips]{epsfig} % um Bilder zu skalieren

\definecolor{darkblue}{rgb}{0,0,.5}
\definecolor{darkmagenta}{rgb}{.5,0,.6}
\usepackage[  %zum einbinden von urls
    bookmarks,
    bookmarksopen=true,
    colorlinks=true,% diese Farbdefinitionen zeichnen Links im PDF farblich aus
    linkcolor=darkblue, %red, % einfache interne Verknüpfungen
    anchorcolor=black,% Ankertext
    citecolor=darkmagenta, % Verweise auf Literaturverzeichniseinträge im Text
    filecolor=magenta, % Verknüpfungen, die lokale Dateien öffnen
    menucolor=red, % Acrobat-Menüpunkte
    urlcolor=blue,
% diese Farbdefinitionen sollten für den Druck verwendet werden (alles schwarz)
    %linkcolor=black, % einfache interne Verknüpfungen
    %anchorcolor=black, % Ankertext
    %citecolor=black, % Verweise auf Literaturverzeichniseinträge im Text
    %filecolor=black, % Verknüpfungen, die lokale Dateien öffnen
    %menucolor=black, % Acrobat-Menüpunkte
    %urlcolor=black,
    backref,
    plainpages=false, % zur korrekten Erstellung der Bookmarks
    pdfpagelabels, % zur korrekten Erstellung der Bookmarks
    hypertexnames=false, % zur korrekten Erstellung der Bookmarks
    linktocpage % Seitenzahlen anstatt Text im Inhaltsverzeichnis verlinken
]{hyperref}


%gute Tabellen
\usepackage{booktabs,colortbl,tabularx}
\usepackage{multirow}

% Einfache Definition der Zeilenabstände und Seitenränder etc. -----------------
\usepackage{setspace}
\usepackage{geometry}


\geometry{paper=a4paper,
left=22mm,
right=18mm,
top=30mm,
bottom=20mm}

%Tabelle feste spaltenbreite mit Ausrichtung rechts, center oder links
\usepackage{tabularx}
\newcolumntype{L}[1]{>{\raggedright\arraybackslash}p{#1}} % linksbündig mit Breitenangabe
\newcolumntype{C}[1]{>{\centering\arraybackslash}p{#1}} % zentriert mit Breitenangabe
\newcolumntype{R}[1]{>{\raggedleft\arraybackslash}p{#1}} % rechtsbündig mit Breitenangabe


% Kopf- und Fußzeilen ----------------------------------------------------------
\usepackage{scrlayer-scrpage}
\pagestyle{scrheadings}
% Kopf- und Fußzeile auch auf Kapitelanfangsseiten
% loescht voreingestellte Stile
\clearscrheadings
\clearscrplain

\renewcommand*{\chapterpagestyle}{scrheadings}
% Schriftform der Kopfzeile
\renewcommand{\headfont}{\normalfont}
\setlength{\headheight}{0mm} % Höhe der Kopfzeile

% Schriftform der Fußzeile
\newcommand*{\totalpagemark}{\thepage/\pageref{LastPage}}
\ofoot{\totalpagemark}
\setlength{\footskip}{12mm}

%Überschriften Anpassen
\renewcommand*{\othersectionlevelsformat}[3]{%
  \makebox[1.5cm][l]{#3\autodot}%
}%
\renewcommand*{\chapterformat}{%
  \makebox[1.5cm][l]{\thechapter\autodot}%
}
\renewcommand{\chapterheadstartvskip}{\vspace*{-4\topskip}}
\renewcommand{\chapterheadendvskip}{\vspace*{0.8\topskip}}

\setlength{\fboxsep}{1.7mm}

%%%%%%%%%%%%%%%%%%%%%%%%%%%%%%%%%%%%%%%%%%%%%%%
% Wir definieren einige Dinge nur einmal und verwenden sie dann mehrfach.
\newcommand*\Title{Mein Kochbuch}
\newcommand*\Subject{}
\newcommand*\Author{Alexander Schäfer}
\newcommand*\Keywords{}

% Wir wollen einen Index:
\usepackage{makeidx}\makeindex

% Wir wollen aktive Links und einige Dokumentinformationen:
\usepackage{hyperref}
\hypersetup{%
  pdftitle={\Title},
  pdfsubject={\Subject},
  pdfauthor={\Author},
  pdfkeywords={\Keywords}
}

\title{\Title\\}
\author{\Large \Author}
\begin{document}
%%%%%%%%%%%%%%%%%%%%%%%%
\maketitle  % erzeugt die Titelseite
\tableofcontents % erzeugt das Inhaltsverzeichnis
%\listoffigures % erzeugt das Abbildungsverzeichnis
%\listoftables % erzeugt das Abbildungsverzeichnis
%%%%%%%%%%%%%%%%%%%%%%%%
\chapter{Fleischgerichte}
\ifoot{\href{http://www.chefkoch.de/rezepte/1832561297156839/Wildschweingulasch-mit-Waldpilzen.html}{http://www.chefkoch.de/rezepte/Wildschweingulasch-mit-Waldpilzen.html}}
\section[Wildschweingulasch mit Waldpilzen]{\leafright\, Wildschweingulasch mit Waldpilzen \,\leafleft}
\begin{minipage}[t]{0.34\textwidth}
\vspace{0pt}\fbox{\includegraphics[width=1.0\textwidth]{./Bilder/wildschweingulasch-waldpilzen.png}}
\vspace{0.5cm}

\begin{small}
\begin{tabular}{R{1.7cm} L{3.7cm} }
\multicolumn{2}{c}{\textbf{Zutaten für 4 Portionen }}\\ \toprule
500 g&	 Gulasch vom Wildschwein\\ \midrule[0.1mm]
2 &	 Zwiebeln\\ \midrule[0.1mm]
2 EL&	 Tomatenmark\\ \midrule[0.1mm]
250 ml&	 Rotwein, trocken\\ \midrule[0.1mm]
400 ml&	 Wildfond\\ \midrule[0.1mm]
400 ml&	 Rinderbrühe, (Instant)\\ \midrule[0.1mm]
500 g&	 Pilze, gemischt, TK; auf gute Qualität achten\\ \midrule[0.1mm]
2 &	 Wacholderbeeren\\ \midrule[0.1mm]
1 	& Piment\\ \midrule[0.1mm]
2 	& Lorbeerblätter\\ \midrule[0.1mm]
3/4 Becher&	 Crème fraîche\\ \midrule[0.1mm]
 &	 Salz und Pfeffer\\ \bottomrule
\end{tabular}
\end{small}
\end{minipage}
\hfill
\begin{minipage}[t]{0.58\textwidth}
\vspace{0pt}
\subsection*{Zubereitung}
\begin{enumerate}[leftmargin=*, itemindent=14pt]
\item Das Gulasch waschen, trocken tupfen und mit Salz und Pfeffer würzen. Die Zwiebeln in feine Ringe schneiden.

\item Das Gulasch in etwas Öl oder Schmalz rundherum anbraten bis er eine schöne Farbe hat. Dann herausnehmen und abgedeckt zur Seite stellen. Die Zwiebelringe im gleichen Fett anbraten bis sie Farbe genommen haben. Die 2 Löffel Tomatenmark dazugeben und kurz mitbraten. Mit dem Rotwein ablöschen und aufkochen, den Wildfond angießen und aufkochen und dann mit der Brühe ergänzen. Wacholderbeeren, Lorbeerblätter und Piment dazugeben und mit geschlossem Deckel 1 Stunde und 45 Minuten schmoren lassen. Gelegentlich umrühren.

\item Ca. 15 Minuten bevor die Zeit um ist, die Waldpilze tiefgefroren in eine zweite Pfanne geben und auf großer Hitze braten bis das austretende Wasser vollständig verdampft ist. Mit Salz und Pfeffer würzen, mit etwas Wasser oder Brühe ablöschen und zu dem Wildschweingulasch geben.

\item Zusammen nochmal 10 Minuten weiterköcheln. Dann Creme fraiche einrühren, Lorbeerblätter, Wacholderbeeren und Piment entfernen und auf gewünschte Konsistenz eindicken. Falls notwendig (in der Regel nicht) mit Salz und Pfeffer würzen.

\end{enumerate}
Dazu passen Spätzle und evtl. Rotkraut.
\end{minipage}
\vfill
\decothreeright \, \textbf{Arbeitszeit:} 15 Min. / \textbf{Schwierigkeitsgrad:} normal \decothreeleft \hfill Bewertung:  \CIRCLE \CIRCLE \CIRCLE \CIRCLE \CIRCLE
\input{./Rezepte/Lasagne_Bolognese.tex}
\input{./Rezepte/bolognese.tex}
\input{./Rezepte/Gefuellte_Paprika.tex}
\input{./Rezepte/Roemische_Zucchini.tex}
\input{./Rezepte/Erbseneintopf.tex}
\input{./Rezepte/schaschlik_wie_im_kaukasus.tex}
\input{./Rezepte/Fladenbrotburger.tex}
\input{./Rezepte/das_schnellste_huhn.tex}
\input{./Rezepte/Afrikanische_Erdnusssauce.tex}
\input{./Rezepte/toast_hawaii.tex}
\input{./Rezepte/Hähnchenbrustfilet_gewickelt_Bohnen.tex}
\input{./Rezepte/gefuellte_Zucchini.tex}
\input{./Rezepte/Kabeljaufilet_auf_gruenem_spargel.tex}
\input{./Rezepte/Rinderrouladen_klassisch}
\ifoot{\href{https://www.reddit.com/r/Cooking/comments/as9g4u/recipe_lemon_chicken_the_original_cantonese_style/}{https://www.reddit.com/r/}}
\section[Zitronenhähnchen Kantonesischer Stil]{\leafright\, Zitronenhähnchen Kantonesischer Stil\leafleft}
\begin{minipage}[t]{0.34\textwidth}
\vspace{0pt}
\fbox{\includegraphics[width=1\linewidth]{./Bilder/Zitronenhaehnchen_Kantonesischer_Stil.png}}
\vspace{0.5cm}

\begin{small}
\begin{tabular}{R{1.6cm} L{3.8cm} }
\multicolumn{2}{c}{\textbf{Zutaten für 2 Portionen }}\\ \toprule

250 g &	Boneless, skinless chicken breast. Alternatives deboned chicken or duck breast\\ \midrule[0.1mm]
1/2 & Lemon, zested and juiced\\ \midrule[0.1mm]
1/2 & tsp Salt\\ \midrule[0.1mm]
1/2 & tsp Sugar\\ \midrule[0.1mm]
1/4 & tsp liaojiu a.k.a. Shaoxing win\\ \midrule[0.1mm]
1 & tsp light soy sauce\\ \midrule[0.1mm]
Coating  for the chicken\\ \midrule[0.1mm]
cornstarch\\ \midrule[0.1mm]
& 1 egg\\ \midrule[0.1mm]
Sauce\\ \midrule[0.1mm]
50 mL & water\\ \midrule[0.1mm]
1 & tsp stock concentrate (Hühnersuppe)-or- homemade stock\\ \midrule[0.1mm]
1& /4 tsp salt\\ \midrule[0.1mm]
2 & tbsp sugar\\ \midrule[0.1mm]
1& /4 tsp instant custard powder (Puddingpulver) -or- milk powder\\ \midrule[0.1mm]
2 & cloves Galic. Gently smashed. Fried when making the sauce.\\ \midrule[0.1mm]
1 & inch Ginger\\ \bottomrule

\end{tabular}
\end{small}
\end{minipage}
\hfill
\begin{minipage}[t]{0.58\textwidth}
\vspace{0pt}
\subsection*{Zubereitung}
\begin{enumerate}[leftmargin=*, itemindent=14pt]
\item Die Rinderrouladen aufrollen, waschen und mit Küchenkrepp trockentupfen. Zwiebeln in Halbmonde, Gurken in Längsstreifen schneiden, Schere und Küchengarn bereitstellen. 

\item Die ausgebreiteten Rouladen dünn mit Senf bestreichen, salzen und pfeffern, auf jede Roulade mittig in der Länge ca. 1/2 Zwiebel und 1 1/2 Scheiben Frühstücksspeck sowie 1/2 (evtl. mehr) Gurke verteilen. Nun von beiden Längsseiten etwas einschlagen, dann aufrollen und mit dem Küchengarn wie ein Postpaket verschnüren.

\item In einer Pfanne das Butterschmalz heiß werden lassen und die Rouladen dann rundherum darin anbraten, herausnehmen und in einen Schmortopf umfüllen.

\item Den Sellerie, die restliche Zwiebel, das Lauch und die Möhren kleinschneiden und in der Pfanne anbraten. Sobald sie halbwegs "blond" sind, kurz rühren, eine sehr dünne Schicht vom Rotwein angießen, nicht mehr rühren und die Flüssigkeit verdampfen lassen. Sobald das Gemüse dann wieder trockenbrät, wieder eine Schicht angießen, kurz rühren und weiter verdampfen lassen. Dies wiederholen, bis die 1/2 Flasche Wein aufgebraucht ist. Auf diese Art wird das Röstgemüse sehr braun (gut für den Geschmack und die Farbe der Soße) aber nicht trocken. Am Schluss mit der Fleischbrühe, etwas Salz und Pfeffer und einem guten Schuss Gurkensud auffüllen und dann in den Schmortopf zu den Rouladen geben. Den Topf entweder auf kleiner Flamme oder bei ca. 160 Grad im Backofen für 1 1/2 Stunden schmoren lassen. Ab und zu evtl. etwas Flüssigkeit zugießen.

\item Nach 1 1/2 Stunden testen, ob die Rouladen weich sind (einfach mal mit den Kochlöffel ein bisschen draufdrücken, sie sollten sich willig eindrücken lassen, wenn nicht, nochmal eine halbe Stunde weiterschmoren) und dann vorsichtig aus dem Topf heben, warm stellen.

\item Die Soße durch ein Sieb geben, aufkochen. Ca. 1 El Senf mit etwas Wasser und der Speisestärke gut verrühren und in die kochende Soße nach und nach unter Rühren eingießen, bis die gewünschte Konsistenz erreicht ist. Die Soße eventuell nochmal mit Salz, Pfeffer, Rotwein, Gurkensud abschmecken. 

\end{enumerate}
\end{minipage}
\vfill
\decothreeright \, \textbf{Arbeitszeit:} ca. 1 Std. / \textbf{Koch-Backzeit:} ca. 2 Std. /\textbf{Schwierigkeitsgrad:} pfiffig \decothreeleft \hfill \\ Bewertung:  \Circle  \Circle \Circle \Circle \Circle




\chapter{Rezepte ohne Fleisch}
\input{./Rezepte/Suesskartoffelsticks_mit_Honig_Sesam_Dip.tex}
%\input{./Rezepte/Zucchini_Wedges_mit_Pesto_Dip.tex}
\input{./Rezepte/Brennessel-Kartoffel-Suppe.tex}
\input{./Rezepte/Linsen-und-Tomatensuppe.tex}
\input{./Rezepte/Blumenkohl_gebraten.tex}
\input{./Rezepte/Erbsensuppe.tex}
\input{./Rezepte/Linsenbolognese_mit_Pasta.tex}
\input{./Rezepte/Schnelle_Gemuese_Burritos.tex}
\input{./Rezepte/Karotten_Curry_Sugo.tex}
\input{./Rezepte/pikante_reistorte.tex}
\input{./Rezepte/bunte_frittata.tex}
\input{./Rezepte/Rote-Bete-Salat-Erdnussbutter.tex}
\input{./Rezepte/Brokkoli_Salat.tex}

\chapter{Grillrezepte}
\input{./Rezepte/Baguette.tex}
\input{./Rezepte/Mais_Tortiallas.tex}

\chapter{Fermentiertes}
\ifoot{\href{https://www.kochbar.de/rezept/265348/Beilage-Sauerkraut-selbermachen.html/}{https://www.kochbar.de/rezept/265348/Beilage-Sauerkraut-selbermachen.html/}}
\section[Sauerkraut]{\leafright\, Sauerkraut \leafleft}
\begin{minipage}[t]{0.34\textwidth}
\vspace{0pt}
\fbox{\includegraphics[width=1\linewidth]{./Bilder/sauerkraut-selbermachen-rezept.png}}
\vspace{0.5cm}

\begin{small}
\begin{tabular}{R{1.6cm} L{3.8cm} }
\multicolumn{2}{c}{\textbf{Zutaten für 10 Liter }}\\ \toprule
8kg & Weißkraut / Kohlkopf\\ \midrule[0.1mm]
120g& Salz (15 g pro kg Kraut\\ \bottomrule
\end{tabular}
\end{small}
\end{minipage}
\hfill
\begin{minipage}[t]{0.58\textwidth}
\vspace{0pt}
\subsection*{Zubereitung}
\begin{enumerate}[leftmargin=*, itemindent=14pt]


\item Tontopf mit heißem Wasser ausbrühen und dann an der Luft trocknen. Wenn der Tontopf trocken ist, dann legen Sie ihn mit sauberen Weißkohl-Blättern aus. Dadurch bleibt das Sauerkraut von der Farbe her schön hell. Anschließend beginnen Sie nun mit dem eigentlichen Sauerkraut machen.

\item Zuerst entfernen Sie die äußeren grünen Blätter schneiden den Kohlkopf in der Mitte durch und entfernen den dicken Strunk.

\item Nun geht es mit dem Krauthobel weiter. In einer großen Schüssel oder Kunstoffbadewanne sammeln Sie das geraspelte Kraut. Dieses durchmengen Sie im nächsten Schritt mit Salz, das ist wichtig um zu Beginn der Gärung unerwünschte Mikroben fernzuhalten. Ca. 15 Gramm Salz auf 1kg Kohl.

\item Dieses Gemenge kräftig mit den Händen quetschen oder mit den Fäusten stampfen bis sich ordentlich Brühe bildet. Dann in den Tontopf, ca. ¾ voll, einfüllen. Festdrücken, bis ca. 2 cm Brühe über dem Kraut stehen. Nun werden einige Krautblätter als Abschluß aufgelegt.

\item Danach werden auf das Sauerkraut die Abschlußsteine gelegt und eventuell noch mit einem faustgroßen Kieselstein beschwert. Dies ist nötig, damit das Sauerkraut immer in der Lake bleibt und nicht aufquellen kann. Der Tontopf hat einen zusätzlichen Rand, in dem der Deckel liegt. Dort wird Wasser eingefüllt. So können von außen keine Keime eindringen. Die Gase können darüber leicht entweichen

\item 14 Tage bis 3 Wochen bleibt der Topf nun in einem etwas wärmeren Raum stehen, dann erst kommt er in den Keller. Nach weiteren 1 – 2 Wochen kann man ja mal probieren, ob es schon genügend gesäuert hat.
\end{enumerate}

TIPP: In die Tontopfrinne eine starke Salzlake füllen. Es können sich ansonsten in der Rinne Schleimbakterien bilden. Wenn der Topf in einen kühleren Raum gebracht wird, erfolgt beim Temperaturangleich zeitweise ein Sog nach innen. Somit könnten Schleimbakterien ins Kraut gelangen. Meinen ersten Ansatz dieses Jahr konnte ich vermutlich aus diesem Grund entsorgen.
    
\end{minipage}
\vfill
\decothreeright \, \textbf{Arbeitszeit:} - / \textbf{Schwierigkeitsgrad:} leicht \decothreeleft \hfill Bewertung: \Circle  \Circle \Circle  \Circle \Circle


\chapter{Hinweise}
\ifoot{\href{https://missboulette.wordpress.com/2010/07/26/4-schritte-zum-ideal-gerosteten-sesam/}{https://missboulette.wordpress.com/2010/07/26/4-schritte-zum-ideal-gerosteten-sesam/}}
\section[4 Schritte zum ideal gerösteten Sesam]{\leafright\, 4 Schritte zum ideal gerösteten Sesam \leafleft}
\label{sec:4 Schritte zum ideal gerösteten Sesam}

\begin{minipage}[t]{0.34\textwidth}
\vspace{0pt}\fbox{\includegraphics[width=1\linewidth]{./Bilder/ideal_geroesteter_Sesam.png}}
\vspace{0.5cm}

\begin{small}
\begin{tabular}{R{1.6cm} L{3.8cm} }
\multicolumn{2}{c}{\textbf{Zutaten für 1 Portion}}\\ \toprule
etwa 1 & kleine Kaffetasse
\end{tabular}
\end{small}
\end{minipage}
\hfill
\begin{minipage}[t]{0.58\textwidth}
\vspace{0pt}
\subsection*{Zubereitung}
\begin{enumerate}[leftmargin=*, itemindent=14pt]

\item Erhitze eine beschichtete Pfanne bei mittlerer Hitze auf dem Herd.

\item Den Sesam kurz unter kaltem Wasser abwaschen und im Sieb oder auf einem Tuch abtropfen lassen. Gib den noch feuchten Sesam in die Pfanne. Extra Öl brauchst Du zum Rösten nicht.

\item Röste die Kerne unter ständigem Rühren. Nach etwa 2 bis 3 Minuten beginnen die Kerne zu knistern und durch die Pfanne zu hüpfen.

\item Nimm ab dann immer mal wieder Körner zwischen die Finger und versuche den Sesam zwischen ihnen zu zerreiben. Gelingt das einfach, ist der Sesam fertig geröstet.

\item Fülle den gerösteten Sesam in eine Schüssel um und lasse ihn dort auskühlen. In der heißen Pfanne würde er anderenfalls weiterrösten und verbrennen.

\end{enumerate}

Beim Rösten geht es um Sekunden! Die ideale Röststufe ist kurz vor dem verbrennen der Samen erreicht. Sobald der Zenit erreicht ist fällt die Qualität rasant ab. Erkennen kann man das an übermäßiger Rauchentwicklung, d.h. Fett tritt bereits aus den Körnern aus und verbrennt. Die Körner sind ölig, einige bereits geplatzt. Diese Körner würden nur wenige Tage gut, danach schnell unangenehm ranzig schmecken.

Wird der Sesam zum idealen Zeitpunkt herausgenommen, kann er, trocken, luftdicht und dukel gelagert, einige Monate gut überstehen.

\end{minipage}
\vfill
\decothreeright \, \textbf{Arbeitszeit:} wenige Minuten	 / \textbf{Schwierigkeitsgrad:} simpel	 / \decothreeleft \hfill Bewertung: \Circle  \Circle \Circle  \Circle \Circle

\end{document}
